\documentclass[a4paper,11pt]{article}

% ---------- Encoding & typography ----------
\usepackage[T1]{fontenc}
\usepackage[utf8]{inputenc}
\usepackage{lmodern}
\usepackage[stretch=10]{microtype}
\usepackage{parskip} % no indents, space between paragraphs

% ---------- Math & units ----------
\usepackage{amsmath, amssymb, mathtools, bm}
\usepackage{siunitx}
\sisetup{detect-all=true}

% ---------- Graphics & floats ----------
\usepackage{graphicx}
\usepackage{booktabs}
\usepackage{caption}
\usepackage{subcaption}
\usepackage{xcolor}
\usepackage{float}

% ---------- Links & references ----------
\usepackage[hidelinks]{hyperref}
\usepackage[nameinlink,capitalise]{cleveref}

% ---------- Page layout ----------
\usepackage[margin=1in]{geometry}

% ---------- Handy macros ----------
\newcommand{\R}{\mathbb{R}}
\newcommand{\abs}[1]{\left|#1\right|}

\title{Justification of the Gradient Direction (Orthogonality to Level Sets)}
\author{Group 2 -- Ferschl Martin, Reiter Roman, Zenkic Mirza}
\date{\today}

\begin{document}
\maketitle

\section{Task specification}

We consider a scalar function
\[
  f : \R^2 \to \R, \qquad f(x,y),
\]
which is sufficiently smooth (at least continuously differentiable).
For a fixed value \(c \in \R\), the equation
\[
  f(x,y) = c
\]
defines a curve in the plane (a \emph{level set} or \emph{contour line} of \(f\)).
Under suitable regularity assumptions (e.g.\ \(f_x\) and \(f_y\) not both zero on
the curve), this level set can locally be parameterized as
\[
  \bm{x}(t) = 
  \begin{pmatrix}
    x(t) \\[0.3em]
    y(t)
  \end{pmatrix},
  \qquad t \in I \subset \R,
\]
so that
\[
  f(x(t), y(t)) = c \quad \text{for all } t \in I.
\]

The vector
\[
  \bm{x}'(t) =
  \begin{pmatrix}
    x'(t) \\[0.3em]
    y'(t)
  \end{pmatrix}
\]
is the tangent (velocity) vector to the curve at the point
\(\bm{x}(t) = (x(t),y(t))^\top\). The claim is that the gradient
\[
  \nabla f(x,y) =
  \begin{pmatrix}
    f_x(x,y) \\[0.3em]
    f_y(x,y)
  \end{pmatrix}
\]
is orthogonal to this tangent vector at each point on the level curve.

We prove this using the chain rule.

\section{Chain rule argument}

Let \(f(x,y)\) be \(C^1\) (continuously differentiable), and let
\(\bm{x}(t) = (x(t), y(t))^\top\) be a differentiable parameterization of the
level set \(f(x,y) = c\). By definition of the level set, we have
\[
  f(x(t),y(t)) = c \quad \text{for all } t.
\]
We now differentiate this identity with respect to \(t\).

\subsection*{Step 1: Differentiate the composition}

Consider the composite function
\[
  g(t) := f(x(t),y(t)).
\]
We know that \(g(t) \equiv c\) is constant, hence
\[
  g'(t) = 0 \quad \text{for all } t.
\]

On the other hand, by the (one-dimensional) chain rule applied to the
composition \(f \circ \bm{x}\), we have
\[
  g'(t)
  = \frac{d}{dt} f\bigl(x(t),y(t)\bigr)
  = f_x\bigl(x(t),y(t)\bigr) \, x'(t)
    + f_y\bigl(x(t),y(t)\bigr) \, y'(t),
\]
where \(f_x\) and \(f_y\) denote the partial derivatives of \(f\) with respect
to its first and second argument.

Combining both expressions for \(g'(t)\), we obtain the scalar equation
\[
  f_x\bigl(x(t),y(t)\bigr) \, x'(t)
  + f_y\bigl(x(t),y(t)\bigr) \, y'(t)
  = 0.
\]

\subsection*{Step 2: Interpret as a dot product}

We can rewrite this as a dot product of two vectors in \(\R^2\):
\[
  \nabla f\bigl(x(t),y(t)\bigr) \cdot \bm{x}'(t)
  =
  \begin{pmatrix}
    f_x(x(t),y(t)) \\[0.3em]
    f_y(x(t),y(t))
  \end{pmatrix}
  \cdot
  \begin{pmatrix}
    x'(t) \\[0.3em]
    y'(t)
  \end{pmatrix}
  = 0.
\]
Thus, at every point on the level curve where \(\bm{x}'(t) \neq \bm{0}\), we have
\[
  \nabla f\bigl(x(t),y(t)\bigr) \cdot \bm{x}'(t) = 0,
\]
i.e.\ the gradient vector is orthogonal to the tangent vector.

\section{Geometric interpretation and connection to gradient descent}

The above calculation shows that at any point \((x,y)\) on the level set
\(f(x,y) = c\), the gradient \(\nabla f(x,y)\) is perpendicular to the level
curve passing through that point. Equivalently:

\begin{itemize}
  \item \(\nabla f(x,y)\) points in the direction of \emph{steepest increase}
        of \(f\).
  \item \(-\nabla f(x,y)\) points in the direction of \emph{steepest decrease}
        of \(f\).
  \item The level curve \(f(x,y) = c\) is locally ``flat'' in the sense that
        moving along the curve (i.e.\ along \(\bm{x}'(t)\)) leaves the value of
        \(f\) unchanged; hence there is no component of \(\nabla f\) in the
        tangential direction, only in the normal direction.
\end{itemize}

This orthogonality property justifies the gradient descent step used in
Task~4:
\[
  (x_{k+1},y_{k+1})
  = (x_k,y_k) - \gamma \, \nabla f(x_k,y_k).
\]
At each iteration, we move from the current point in the direction
\(-\nabla f\), i.e.\ orthogonally to the contours of \(f\), in the direction
of steepest decrease of the function value. For convex functions, this
iteration can be shown (under suitable step-size conditions) to converge to
a global minimizer.

\end{document}

